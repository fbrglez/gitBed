%\chapter{About Text Items}
%\label{chap-About}
%\newthought{Text items in this chapter} are divided into two sections: 
%(1) test items from Tufte,
%(2) test items  from other sources.
%
%\newthought{For details} about how figures and tables are being represented,
%see Chapters~\ref{chap-Figures} and~\ref{chap-Tables}.
%
%\section{Text Items Selected from Tufte's Book Template}
\label{chap-About-Tufte}

\newthought{The primary text items} for this section are selections
from Tufte Book Template\sidenote{The Tufte Book Template can be accessed
at~\url{https://github.com/Tufte-LaTeX/tufte-latex}.}
Here, we extracted three sections from the Chapter on
{\em On the Use of the tufte-book Document Class}:
From this template, have extracted three sections
\begin{itemize}
\item Page Layout: Headings
\item Sidenotes
\item References
\end{itemize}
However, in this document, we treat all of these three sections as subsections. 


\subsection{Page Layout: Headings}\label{sec:headings}\index{headings}

\newthought{Tufte's books} include the following heading levels: parts,
chapters,\sidenote{Parts and chapters are defined for the \texttt{tufte-book}
class only.}  sections, subsections, and paragraphs.  Not defined by default
are: sub-subsections and subparagraphs.

\begin{table}[h]
  \begin{center}
    \footnotesize%
    \begin{tabular}{lcr}
      \toprule
      Heading & Style & Size \\
      \midrule
      Part & roman & \measure{24}{36}{40} \\
      Chapter & italic & \measure{20}{30}{40} \\
      Section & italic & \measure{12}{16}{26} \\
      Subsection & italic & \measure{11}{15}{26} \\
      Paragraph & italic & 10/14 \\
      \bottomrule
    \end{tabular}
  \end{center}
  \caption{Heading styles used in \BE.}
  \label{tab:heading-styles}
\end{table}

\paragraph{Paragraph} Paragraph headings (as shown here) are introduced by
italicized text and separated from the main paragraph by a bit of space.

This style provides \textsc{a}- and \textsc{b}-heads (that is,
\Verb|\section| and \Verb|\subsection|), demonstrated above.

If you need more than two levels of section headings, you'll have to define
them yourself at the moment; there are no pre-defined styles for anything below
a \Verb|\subsection|.  As Bringhurst points out in \textit{The Elements of
Typographic Style},\cite{Bringhurst2005} you should ``use as many levels of
headings as you need: no more, and no fewer.''

The \TL classes will emit an error if you try to use
\linebreak\Verb|\subsubsection| and smaller headings.

% let's start a new thought -- a new section
\newthought{In his later books},\cite{Tufte2006} Tufte
starts each section with a bit of vertical space, a non-indented paragraph,
and sets the first few words of the sentence in \textsc{small caps}.  To
accomplish this using this style, use the \doccmddef{newthought} command:
\begin{docspec}
  \doccmd{newthought}\{In his later books\}, Tufte starts\ldots
\end{docspec}


\subsection{Sidenotes}\label{sec:sidenotes}
One of the most prominent and distinctive features of this style is the
extensive use of sidenotes.  There is a wide margin to provide ample room
for sidenotes and small figures.  Any \doccmd{footnote}s will automatically
be converted to sidenotes.\footnote{This is a sidenote that was entered
using the \texttt{\textbackslash footnote} command.}  If you'd like to place ancillary
information in the margin without the sidenote mark (the superscript
number), you can use the \doccmd{marginnote} command.\marginnote{This is a
margin note.  Notice that there isn't a number preceding the note, and
there is no number in the main text where this note was written.}

The specification of the \doccmddef{sidenote} command is:
\begin{docspec}
  \doccmd{sidenote}[\docopt{number}][\docopt{offset}]\{\docarg{Sidenote text.}\}
\end{docspec}

Both the \docopt{number} and \docopt{offset} arguments are optional.  If you
provide a \docopt{number} argument, then that number will be used as the
sidenote number.  It will change of the number of the current sidenote only and
will not affect the numbering sequence of subsequent sidenotes.

Sometimes a sidenote may run over the top of other text or graphics in the
margin space.  If this happens, you can adjust the vertical position of the
sidenote by providing a dimension in the \docopt{offset} argument.  Some
examples of valid dimensions are:
\begin{docspec}
  \ttfamily 1.0in \qquad 2.54cm \qquad 254mm \qquad 6\Verb|\baselineskip|
\end{docspec}
If the dimension is positive it will push the sidenote down the page; if the
dimension is negative, it will move the sidenote up the page.

While both the \docopt{number} and \docopt{offset} arguments are optional, they
must be provided in order.  To adjust the vertical position of the sidenote
while leaving the sidenote number alone, use the following syntax:
\begin{docspec}
  \doccmd{sidenote}[][\docopt{offset}]\{\docarg{Sidenote text.}\}
\end{docspec}
The empty brackets tell the \Verb|\sidenote| command to use the default
sidenote number.

If you \emph{only} want to change the sidenote number, however, you may
completely omit the \docopt{offset} argument:
\begin{docspec}
  \doccmd{sidenote}[\docopt{number}]\{\docarg{Sidenote text.}\}
\end{docspec}

The \doccmddef{marginnote} command has a similar \docarg{offset} argument:
\begin{docspec}
  \doccmd{marginnote}[\docopt{offset}]\{\docarg{Margin note text.}\}
\end{docspec}

\subsection{References}
References are placed alongside their citations as sidenotes,
as well.  This can be accomplished using the normal \doccmddef{cite}
command.\sidenote{The first paragraph of this document includes a citation.}

The complete list of references may also be printed automatically by using
the \doccmddef{bibliography} command.  (See the end of this document for an
example.)  If you do not want to print a bibliography at the end of your
document, use the \doccmddef{nobibliography} command in its place.  

To enter multiple citations at one location,\cite[-3\baselineskip]{Tufte2006,Tufte1990} you can
provide a list of keys separated by commas and the same optional vertical
offset argument: \Verb|\cite{Tufte2006,Tufte1990}|.  
\begin{docspec}
  \doccmd{cite}[\docopt{offset}]\{\docarg{bibkey1,bibkey2,\ldots}\}
\end{docspec}
%%\chapter{About Text Items}
%\label{chap-About}
%\newthought{Text items in this chapter} are divided into two sections: 
%(1) test items from Tufte,
%(2) test items  from other sources.
%
%\newthought{For details} about how figures and tables are being represented,
%see Chapters~\ref{chap-Figures} and~\ref{chap-Tables}.
%
%\section{Text Items Selected from Tufte's Book Template}
\label{chap-About-Tufte}

\newthought{The primary text items} for this section are selections
from Tufte Book Template\sidenote{The Tufte Book Template can be accessed
at~\url{https://github.com/Tufte-LaTeX/tufte-latex}.}
Here, we extracted three sections from the Chapter on
{\em On the Use of the tufte-book Document Class}:
From this template, have extracted three sections
\begin{itemize}
\item Page Layout: Headings
\item Sidenotes
\item References
\end{itemize}
However, in this document, we treat all of these three sections as subsections. 


\subsection{Page Layout: Headings}\label{sec:headings}\index{headings}

\newthought{Tufte's books} include the following heading levels: parts,
chapters,\sidenote{Parts and chapters are defined for the \texttt{tufte-book}
class only.}  sections, subsections, and paragraphs.  Not defined by default
are: sub-subsections and subparagraphs.

\begin{table}[h]
  \begin{center}
    \footnotesize%
    \begin{tabular}{lcr}
      \toprule
      Heading & Style & Size \\
      \midrule
      Part & roman & \measure{24}{36}{40} \\
      Chapter & italic & \measure{20}{30}{40} \\
      Section & italic & \measure{12}{16}{26} \\
      Subsection & italic & \measure{11}{15}{26} \\
      Paragraph & italic & 10/14 \\
      \bottomrule
    \end{tabular}
  \end{center}
  \caption{Heading styles used in \BE.}
  \label{tab:heading-styles}
\end{table}

\paragraph{Paragraph} Paragraph headings (as shown here) are introduced by
italicized text and separated from the main paragraph by a bit of space.

This style provides \textsc{a}- and \textsc{b}-heads (that is,
\Verb|\section| and \Verb|\subsection|), demonstrated above.

If you need more than two levels of section headings, you'll have to define
them yourself at the moment; there are no pre-defined styles for anything below
a \Verb|\subsection|.  As Bringhurst points out in \textit{The Elements of
Typographic Style},\cite{Bringhurst2005} you should ``use as many levels of
headings as you need: no more, and no fewer.''

The \TL classes will emit an error if you try to use
\linebreak\Verb|\subsubsection| and smaller headings.

% let's start a new thought -- a new section
\newthought{In his later books},\cite{Tufte2006} Tufte
starts each section with a bit of vertical space, a non-indented paragraph,
and sets the first few words of the sentence in \textsc{small caps}.  To
accomplish this using this style, use the \doccmddef{newthought} command:
\begin{docspec}
  \doccmd{newthought}\{In his later books\}, Tufte starts\ldots
\end{docspec}


\subsection{Sidenotes}\label{sec:sidenotes}
One of the most prominent and distinctive features of this style is the
extensive use of sidenotes.  There is a wide margin to provide ample room
for sidenotes and small figures.  Any \doccmd{footnote}s will automatically
be converted to sidenotes.\footnote{This is a sidenote that was entered
using the \texttt{\textbackslash footnote} command.}  If you'd like to place ancillary
information in the margin without the sidenote mark (the superscript
number), you can use the \doccmd{marginnote} command.\marginnote{This is a
margin note.  Notice that there isn't a number preceding the note, and
there is no number in the main text where this note was written.}

The specification of the \doccmddef{sidenote} command is:
\begin{docspec}
  \doccmd{sidenote}[\docopt{number}][\docopt{offset}]\{\docarg{Sidenote text.}\}
\end{docspec}

Both the \docopt{number} and \docopt{offset} arguments are optional.  If you
provide a \docopt{number} argument, then that number will be used as the
sidenote number.  It will change of the number of the current sidenote only and
will not affect the numbering sequence of subsequent sidenotes.

Sometimes a sidenote may run over the top of other text or graphics in the
margin space.  If this happens, you can adjust the vertical position of the
sidenote by providing a dimension in the \docopt{offset} argument.  Some
examples of valid dimensions are:
\begin{docspec}
  \ttfamily 1.0in \qquad 2.54cm \qquad 254mm \qquad 6\Verb|\baselineskip|
\end{docspec}
If the dimension is positive it will push the sidenote down the page; if the
dimension is negative, it will move the sidenote up the page.

While both the \docopt{number} and \docopt{offset} arguments are optional, they
must be provided in order.  To adjust the vertical position of the sidenote
while leaving the sidenote number alone, use the following syntax:
\begin{docspec}
  \doccmd{sidenote}[][\docopt{offset}]\{\docarg{Sidenote text.}\}
\end{docspec}
The empty brackets tell the \Verb|\sidenote| command to use the default
sidenote number.

If you \emph{only} want to change the sidenote number, however, you may
completely omit the \docopt{offset} argument:
\begin{docspec}
  \doccmd{sidenote}[\docopt{number}]\{\docarg{Sidenote text.}\}
\end{docspec}

The \doccmddef{marginnote} command has a similar \docarg{offset} argument:
\begin{docspec}
  \doccmd{marginnote}[\docopt{offset}]\{\docarg{Margin note text.}\}
\end{docspec}

\subsection{References}
References are placed alongside their citations as sidenotes,
as well.  This can be accomplished using the normal \doccmddef{cite}
command.\sidenote{The first paragraph of this document includes a citation.}

The complete list of references may also be printed automatically by using
the \doccmddef{bibliography} command.  (See the end of this document for an
example.)  If you do not want to print a bibliography at the end of your
document, use the \doccmddef{nobibliography} command in its place.  

To enter multiple citations at one location,\cite[-3\baselineskip]{Tufte2006,Tufte1990} you can
provide a list of keys separated by commas and the same optional vertical
offset argument: \Verb|\cite{Tufte2006,Tufte1990}|.  
\begin{docspec}
  \doccmd{cite}[\docopt{offset}]\{\docarg{bibkey1,bibkey2,\ldots}\}
\end{docspec}
%%\chapter{About Text Items}
%\label{chap-About}
%\newthought{Text items in this chapter} are divided into two sections: 
%(1) test items from Tufte,
%(2) test items  from other sources.
%
%\newthought{For details} about how figures and tables are being represented,
%see Chapters~\ref{chap-Figures} and~\ref{chap-Tables}.
%
%\section{Text Items Selected from Tufte's Book Template}
\label{chap-About-Tufte}

\newthought{The primary text items} for this section are selections
from Tufte Book Template\sidenote{The Tufte Book Template can be accessed
at~\url{https://github.com/Tufte-LaTeX/tufte-latex}.}
Here, we extracted three sections from the Chapter on
{\em On the Use of the tufte-book Document Class}:
From this template, have extracted three sections
\begin{itemize}
\item Page Layout: Headings
\item Sidenotes
\item References
\end{itemize}
However, in this document, we treat all of these three sections as subsections. 


\subsection{Page Layout: Headings}\label{sec:headings}\index{headings}

\newthought{Tufte's books} include the following heading levels: parts,
chapters,\sidenote{Parts and chapters are defined for the \texttt{tufte-book}
class only.}  sections, subsections, and paragraphs.  Not defined by default
are: sub-subsections and subparagraphs.

\begin{table}[h]
  \begin{center}
    \footnotesize%
    \begin{tabular}{lcr}
      \toprule
      Heading & Style & Size \\
      \midrule
      Part & roman & \measure{24}{36}{40} \\
      Chapter & italic & \measure{20}{30}{40} \\
      Section & italic & \measure{12}{16}{26} \\
      Subsection & italic & \measure{11}{15}{26} \\
      Paragraph & italic & 10/14 \\
      \bottomrule
    \end{tabular}
  \end{center}
  \caption{Heading styles used in \BE.}
  \label{tab:heading-styles}
\end{table}

\paragraph{Paragraph} Paragraph headings (as shown here) are introduced by
italicized text and separated from the main paragraph by a bit of space.

This style provides \textsc{a}- and \textsc{b}-heads (that is,
\Verb|\section| and \Verb|\subsection|), demonstrated above.

If you need more than two levels of section headings, you'll have to define
them yourself at the moment; there are no pre-defined styles for anything below
a \Verb|\subsection|.  As Bringhurst points out in \textit{The Elements of
Typographic Style},\cite{Bringhurst2005} you should ``use as many levels of
headings as you need: no more, and no fewer.''

The \TL classes will emit an error if you try to use
\linebreak\Verb|\subsubsection| and smaller headings.

% let's start a new thought -- a new section
\newthought{In his later books},\cite{Tufte2006} Tufte
starts each section with a bit of vertical space, a non-indented paragraph,
and sets the first few words of the sentence in \textsc{small caps}.  To
accomplish this using this style, use the \doccmddef{newthought} command:
\begin{docspec}
  \doccmd{newthought}\{In his later books\}, Tufte starts\ldots
\end{docspec}


\subsection{Sidenotes}\label{sec:sidenotes}
One of the most prominent and distinctive features of this style is the
extensive use of sidenotes.  There is a wide margin to provide ample room
for sidenotes and small figures.  Any \doccmd{footnote}s will automatically
be converted to sidenotes.\footnote{This is a sidenote that was entered
using the \texttt{\textbackslash footnote} command.}  If you'd like to place ancillary
information in the margin without the sidenote mark (the superscript
number), you can use the \doccmd{marginnote} command.\marginnote{This is a
margin note.  Notice that there isn't a number preceding the note, and
there is no number in the main text where this note was written.}

The specification of the \doccmddef{sidenote} command is:
\begin{docspec}
  \doccmd{sidenote}[\docopt{number}][\docopt{offset}]\{\docarg{Sidenote text.}\}
\end{docspec}

Both the \docopt{number} and \docopt{offset} arguments are optional.  If you
provide a \docopt{number} argument, then that number will be used as the
sidenote number.  It will change of the number of the current sidenote only and
will not affect the numbering sequence of subsequent sidenotes.

Sometimes a sidenote may run over the top of other text or graphics in the
margin space.  If this happens, you can adjust the vertical position of the
sidenote by providing a dimension in the \docopt{offset} argument.  Some
examples of valid dimensions are:
\begin{docspec}
  \ttfamily 1.0in \qquad 2.54cm \qquad 254mm \qquad 6\Verb|\baselineskip|
\end{docspec}
If the dimension is positive it will push the sidenote down the page; if the
dimension is negative, it will move the sidenote up the page.

While both the \docopt{number} and \docopt{offset} arguments are optional, they
must be provided in order.  To adjust the vertical position of the sidenote
while leaving the sidenote number alone, use the following syntax:
\begin{docspec}
  \doccmd{sidenote}[][\docopt{offset}]\{\docarg{Sidenote text.}\}
\end{docspec}
The empty brackets tell the \Verb|\sidenote| command to use the default
sidenote number.

If you \emph{only} want to change the sidenote number, however, you may
completely omit the \docopt{offset} argument:
\begin{docspec}
  \doccmd{sidenote}[\docopt{number}]\{\docarg{Sidenote text.}\}
\end{docspec}

The \doccmddef{marginnote} command has a similar \docarg{offset} argument:
\begin{docspec}
  \doccmd{marginnote}[\docopt{offset}]\{\docarg{Margin note text.}\}
\end{docspec}

\subsection{References}
References are placed alongside their citations as sidenotes,
as well.  This can be accomplished using the normal \doccmddef{cite}
command.\sidenote{The first paragraph of this document includes a citation.}

The complete list of references may also be printed automatically by using
the \doccmddef{bibliography} command.  (See the end of this document for an
example.)  If you do not want to print a bibliography at the end of your
document, use the \doccmddef{nobibliography} command in its place.  

To enter multiple citations at one location,\cite[-3\baselineskip]{Tufte2006,Tufte1990} you can
provide a list of keys separated by commas and the same optional vertical
offset argument: \Verb|\cite{Tufte2006,Tufte1990}|.  
\begin{docspec}
  \doccmd{cite}[\docopt{offset}]\{\docarg{bibkey1,bibkey2,\ldots}\}
\end{docspec}
%%\chapter{About Text Items}
%\label{chap-About}
%\newthought{Text items in this chapter} are divided into two sections: 
%(1) test items from Tufte,
%(2) test items  from other sources.
%
%\newthought{For details} about how figures and tables are being represented,
%see Chapters~\ref{chap-Figures} and~\ref{chap-Tables}.
%
%\input{Ch-About-Tufte}
%\input{Ch-About-Other}
\section{About Text Items from Other Sources}
\label{chap-About-Other}

\newthought{The primary text items} for this section are selections
from\cite[-1ex]{Lib-OPUS2-labs-2015-arxiv-Boskovic}.

\vspace*{1.5ex}\noindent
The command \verb+\cmt+ as listed below,
\cmt{creates a `comment' sentence like this one.}\\  
\verb+\newcommand{\cmt}[1]{\textsf{[#1]}}+

\vspace*{1.5ex}\noindent
The command \verb+\OMIT+ as listed below\\
\verb+\newcommand{\OMIT}[1]{}+ \\
suppresses a block of text listed in the latex source code below
(it makes it invisible).
\OMIT{
\TOPIC{Local paragraph} 
This boldface {\em Local paragraph.} has been created by using 
the local command \verb+\TOPIC+: see the command below\\
\verb+\newcommand{\TOPIC}[1]{\vspace{1.3ex}\par\noindent\textbf{#1.}}+
}

\TOPIC{Local paragraph} 
This boldface headings, terminated with a period, {\em Local paragraph.} has been created by using 
the local command \verb+\TOPIC+: see the command below\\
\verb+\newcommand{\TOPIC}[1]{\vspace{1.3ex}\par\noindent\textbf{#1.}}+\\
\noindent
NOTE: such heading may be considered `too bold' in the context of Tufte's ideas in
{\em Beautiful Evidence}.

\paragraph{Tufte's Paragraph} Paragraph headings (as shown here) are introduced by
italicized text and separated from the main paragraph by a bit of space.
The command is \\
\verb+\paragraph{}+


\TOPIC{About the labs problem}
The {\em aperiodic low-autocorrelation binary sequence} (\labs) problem 
has a simple formulation: take a binary sequence of length $L$,  
$S = s_1 s_2 \ldots s_L$, $s_i \in \{ +1,-1 \}$, the autocorrelation function
$C_k(S) = \sum_{i=1}^{L-k}s_{i}s_{i+k},$ and minimize the energy function:
\begin{equation}
E(S) =  \sum_{k=1}^{L-1}C_{k}^{2}(S)
\label{eq_energy}
\end{equation}
or alternatively, maximize the 
{\em merit factor F}\cite[-1ex]{Lib-OPUS-labs-1977-IEEE_TIT-Golay,
Lib-OPUS-labs-1982-IEEE_TIT-Golay,
Lib-OPUS-labs-1990-IEEE_TIT-Golay-skewsym}:
\begin{equation}
F(S) =  {L^2}/({2E(S)}).
\label{eq_meritFactor}
\end{equation}
 

The asymptotic value for the maximum merit factor $F$, introduced
by Golay, has been re-derived
using arguments from statistical mechanics\cite[2ex]{Lib-OPUS-labs-1987-JourPhys-Bernasconi}:
\begin{equation}
{\rm as~~~} L \rightarrow \infty {\rm ,~~~then~~~}F  \rightarrow 12.3248
\label{eq_asymptotic}
\end{equation} 
The publication of the asymptotic value in Eq. \ref{eq_asymptotic}
is providing an on-going challenge since no published solutions can yet claim to converge to this value
as the length of the sequence increases. 

\TOPIC{Creating a filler text}
The remainder of this paragraph has been created with the command \verb+\lipsum[4]+.
\lipsum[4]

%\paragraph{Creating more filler text}
%The remainder of this paragraph has been created with command \verb+\lipsum[1]+.
%\lipsum[1]

\TOPIC{More about the labs problem}
Finding the optimum sequence is significantly harder than solving the special cases of the Ising spin-glass problems with limited interaction and periodic boundary conditions, for 
example\cite{Lib-OPUS-labs-2003-GECCO-Goldberg-periodic}. 
While effective methods have been presented to solve the special 
cases\cite[3ex]{Lib-OPUS-labs-2003-GECCO-Goldberg-periodic}, up to $L = 400$, 
the best merit factors that has also been {\em proven optimal} for the problem as formulated in Eq. \ref{eq_meritFactor}
are presently known for values of $L \le 60$ 
only\cite[3ex]{Lib-OPUS-labs-1996-JPhysA-Mertens-BB_solutions}.
A web page of \labs\ best merit factors and solutions, 
up to the sequence length of $L=304$, has been compiled by Joshua Knauer in 2002.
This page is no longer accessible and has now been restored 
at two mirroring sites\cite[3ex]{Lib-OPUS2-labs-2014-homepage-Knauer} next to 
additional and  comprehensive tables of {\em best-value solutions}.
These tables contain not only updates on the best known figures of merit but also 
on the number of {\em unique}  solutions in {\em canonic form} and the solutions themselves.
%

{\bf Now, we need more text on this page if we are to make
extra space for the citation that should be moved into the margins on the next page.}
Can or should we have {\em some citations} not appear in the margin, only at
under Bibliography at the very end? Such a feature would not appear logical
in the context of Tufte's book template, would it?
The message from Tufte's book template seem to be: do not overcrowd with citations on any given page,
have sufficient text to justify the introduction (and context) of any new citation ... This criterion can be considered different for books when compared to peer-reviewed article ...


This text has been created with the command \verb+\lipsum[4]+.
\lipsum[1]

Relationships between results reported 
in\cite[-5ex]{Lib-OPUS-labs-1982-IEEE_TIT-Golay,Lib-OPUS-labs-1985-Phillips-Beenker}
and\cite[3ex]{Lib-OPUS-labs-1987-JourPhys-Bernasconi,Lib-OPUS-labs-1990-IEEE_TIT-Golay-skewsym}, 
and all subsequent updates under\cite[3ex]{Lib-OPUS2-labs-2014-homepage}
are depicted in four panels in Figure~\ref{fg-R-labs-wide-4-figures} (See Chapter~\ref{chap-Figures}.
The latest experimental results support the trend towards the conjectured asymptotic value of $F=12.3248$,
however as we demonstrate later on in the paper, the computational cost to reach this value may well exceed the currently available resources unless a better solver is discovered.


\section{About Text Items from Other Sources}
\label{chap-About-Other}

\newthought{The primary text items} for this section are selections
from\cite[-1ex]{Lib-OPUS2-labs-2015-arxiv-Boskovic}.

\vspace*{1.5ex}\noindent
The command \verb+\cmt+ as listed below,
\cmt{creates a `comment' sentence like this one.}\\  
\verb+\newcommand{\cmt}[1]{\textsf{[#1]}}+

\vspace*{1.5ex}\noindent
The command \verb+\OMIT+ as listed below\\
\verb+\newcommand{\OMIT}[1]{}+ \\
suppresses a block of text listed in the latex source code below
(it makes it invisible).
\OMIT{
\TOPIC{Local paragraph} 
This boldface {\em Local paragraph.} has been created by using 
the local command \verb+\TOPIC+: see the command below\\
\verb+\newcommand{\TOPIC}[1]{\vspace{1.3ex}\par\noindent\textbf{#1.}}+
}

\TOPIC{Local paragraph} 
This boldface headings, terminated with a period, {\em Local paragraph.} has been created by using 
the local command \verb+\TOPIC+: see the command below\\
\verb+\newcommand{\TOPIC}[1]{\vspace{1.3ex}\par\noindent\textbf{#1.}}+\\
\noindent
NOTE: such heading may be considered `too bold' in the context of Tufte's ideas in
{\em Beautiful Evidence}.

\paragraph{Tufte's Paragraph} Paragraph headings (as shown here) are introduced by
italicized text and separated from the main paragraph by a bit of space.
The command is \\
\verb+\paragraph{}+


\TOPIC{About the labs problem}
The {\em aperiodic low-autocorrelation binary sequence} (\labs) problem 
has a simple formulation: take a binary sequence of length $L$,  
$S = s_1 s_2 \ldots s_L$, $s_i \in \{ +1,-1 \}$, the autocorrelation function
$C_k(S) = \sum_{i=1}^{L-k}s_{i}s_{i+k},$ and minimize the energy function:
\begin{equation}
E(S) =  \sum_{k=1}^{L-1}C_{k}^{2}(S)
\label{eq_energy}
\end{equation}
or alternatively, maximize the 
{\em merit factor F}\cite[-1ex]{Lib-OPUS-labs-1977-IEEE_TIT-Golay,
Lib-OPUS-labs-1982-IEEE_TIT-Golay,
Lib-OPUS-labs-1990-IEEE_TIT-Golay-skewsym}:
\begin{equation}
F(S) =  {L^2}/({2E(S)}).
\label{eq_meritFactor}
\end{equation}
 

The asymptotic value for the maximum merit factor $F$, introduced
by Golay, has been re-derived
using arguments from statistical mechanics\cite[2ex]{Lib-OPUS-labs-1987-JourPhys-Bernasconi}:
\begin{equation}
{\rm as~~~} L \rightarrow \infty {\rm ,~~~then~~~}F  \rightarrow 12.3248
\label{eq_asymptotic}
\end{equation} 
The publication of the asymptotic value in Eq. \ref{eq_asymptotic}
is providing an on-going challenge since no published solutions can yet claim to converge to this value
as the length of the sequence increases. 

\TOPIC{Creating a filler text}
The remainder of this paragraph has been created with the command \verb+\lipsum[4]+.
\lipsum[4]

%\paragraph{Creating more filler text}
%The remainder of this paragraph has been created with command \verb+\lipsum[1]+.
%\lipsum[1]

\TOPIC{More about the labs problem}
Finding the optimum sequence is significantly harder than solving the special cases of the Ising spin-glass problems with limited interaction and periodic boundary conditions, for 
example\cite{Lib-OPUS-labs-2003-GECCO-Goldberg-periodic}. 
While effective methods have been presented to solve the special 
cases\cite[3ex]{Lib-OPUS-labs-2003-GECCO-Goldberg-periodic}, up to $L = 400$, 
the best merit factors that has also been {\em proven optimal} for the problem as formulated in Eq. \ref{eq_meritFactor}
are presently known for values of $L \le 60$ 
only\cite[3ex]{Lib-OPUS-labs-1996-JPhysA-Mertens-BB_solutions}.
A web page of \labs\ best merit factors and solutions, 
up to the sequence length of $L=304$, has been compiled by Joshua Knauer in 2002.
This page is no longer accessible and has now been restored 
at two mirroring sites\cite[3ex]{Lib-OPUS2-labs-2014-homepage-Knauer} next to 
additional and  comprehensive tables of {\em best-value solutions}.
These tables contain not only updates on the best known figures of merit but also 
on the number of {\em unique}  solutions in {\em canonic form} and the solutions themselves.
%

{\bf Now, we need more text on this page if we are to make
extra space for the citation that should be moved into the margins on the next page.}
Can or should we have {\em some citations} not appear in the margin, only at
under Bibliography at the very end? Such a feature would not appear logical
in the context of Tufte's book template, would it?
The message from Tufte's book template seem to be: do not overcrowd with citations on any given page,
have sufficient text to justify the introduction (and context) of any new citation ... This criterion can be considered different for books when compared to peer-reviewed article ...


This text has been created with the command \verb+\lipsum[4]+.
\lipsum[1]

Relationships between results reported 
in\cite[-5ex]{Lib-OPUS-labs-1982-IEEE_TIT-Golay,Lib-OPUS-labs-1985-Phillips-Beenker}
and\cite[3ex]{Lib-OPUS-labs-1987-JourPhys-Bernasconi,Lib-OPUS-labs-1990-IEEE_TIT-Golay-skewsym}, 
and all subsequent updates under\cite[3ex]{Lib-OPUS2-labs-2014-homepage}
are depicted in four panels in Figure~\ref{fg-R-labs-wide-4-figures} (See Chapter~\ref{chap-Figures}.
The latest experimental results support the trend towards the conjectured asymptotic value of $F=12.3248$,
however as we demonstrate later on in the paper, the computational cost to reach this value may well exceed the currently available resources unless a better solver is discovered.


\section{About Text Items from Other Sources}
\label{chap-About-Other}

\newthought{The primary text items} for this section are selections
from\cite[-1ex]{Lib-OPUS2-labs-2015-arxiv-Boskovic}.

\vspace*{1.5ex}\noindent
The command \verb+\cmt+ as listed below,
\cmt{creates a `comment' sentence like this one.}\\  
\verb+\newcommand{\cmt}[1]{\textsf{[#1]}}+

\vspace*{1.5ex}\noindent
The command \verb+\OMIT+ as listed below\\
\verb+\newcommand{\OMIT}[1]{}+ \\
suppresses a block of text listed in the latex source code below
(it makes it invisible).
\OMIT{
\TOPIC{Local paragraph} 
This boldface {\em Local paragraph.} has been created by using 
the local command \verb+\TOPIC+: see the command below\\
\verb+\newcommand{\TOPIC}[1]{\vspace{1.3ex}\par\noindent\textbf{#1.}}+
}

\TOPIC{Local paragraph} 
This boldface headings, terminated with a period, {\em Local paragraph.} has been created by using 
the local command \verb+\TOPIC+: see the command below\\
\verb+\newcommand{\TOPIC}[1]{\vspace{1.3ex}\par\noindent\textbf{#1.}}+\\
\noindent
NOTE: such heading may be considered `too bold' in the context of Tufte's ideas in
{\em Beautiful Evidence}.

\paragraph{Tufte's Paragraph} Paragraph headings (as shown here) are introduced by
italicized text and separated from the main paragraph by a bit of space.
The command is \\
\verb+\paragraph{}+


\TOPIC{About the labs problem}
The {\em aperiodic low-autocorrelation binary sequence} (\labs) problem 
has a simple formulation: take a binary sequence of length $L$,  
$S = s_1 s_2 \ldots s_L$, $s_i \in \{ +1,-1 \}$, the autocorrelation function
$C_k(S) = \sum_{i=1}^{L-k}s_{i}s_{i+k},$ and minimize the energy function:
\begin{equation}
E(S) =  \sum_{k=1}^{L-1}C_{k}^{2}(S)
\label{eq_energy}
\end{equation}
or alternatively, maximize the 
{\em merit factor F}\cite[-1ex]{Lib-OPUS-labs-1977-IEEE_TIT-Golay,
Lib-OPUS-labs-1982-IEEE_TIT-Golay,
Lib-OPUS-labs-1990-IEEE_TIT-Golay-skewsym}:
\begin{equation}
F(S) =  {L^2}/({2E(S)}).
\label{eq_meritFactor}
\end{equation}
 

The asymptotic value for the maximum merit factor $F$, introduced
by Golay, has been re-derived
using arguments from statistical mechanics\cite[2ex]{Lib-OPUS-labs-1987-JourPhys-Bernasconi}:
\begin{equation}
{\rm as~~~} L \rightarrow \infty {\rm ,~~~then~~~}F  \rightarrow 12.3248
\label{eq_asymptotic}
\end{equation} 
The publication of the asymptotic value in Eq. \ref{eq_asymptotic}
is providing an on-going challenge since no published solutions can yet claim to converge to this value
as the length of the sequence increases. 

\TOPIC{Creating a filler text}
The remainder of this paragraph has been created with the command \verb+\lipsum[4]+.
\lipsum[4]

%\paragraph{Creating more filler text}
%The remainder of this paragraph has been created with command \verb+\lipsum[1]+.
%\lipsum[1]

\TOPIC{More about the labs problem}
Finding the optimum sequence is significantly harder than solving the special cases of the Ising spin-glass problems with limited interaction and periodic boundary conditions, for 
example\cite{Lib-OPUS-labs-2003-GECCO-Goldberg-periodic}. 
While effective methods have been presented to solve the special 
cases\cite[3ex]{Lib-OPUS-labs-2003-GECCO-Goldberg-periodic}, up to $L = 400$, 
the best merit factors that has also been {\em proven optimal} for the problem as formulated in Eq. \ref{eq_meritFactor}
are presently known for values of $L \le 60$ 
only\cite[3ex]{Lib-OPUS-labs-1996-JPhysA-Mertens-BB_solutions}.
A web page of \labs\ best merit factors and solutions, 
up to the sequence length of $L=304$, has been compiled by Joshua Knauer in 2002.
This page is no longer accessible and has now been restored 
at two mirroring sites\cite[3ex]{Lib-OPUS2-labs-2014-homepage-Knauer} next to 
additional and  comprehensive tables of {\em best-value solutions}.
These tables contain not only updates on the best known figures of merit but also 
on the number of {\em unique}  solutions in {\em canonic form} and the solutions themselves.
%

{\bf Now, we need more text on this page if we are to make
extra space for the citation that should be moved into the margins on the next page.}
Can or should we have {\em some citations} not appear in the margin, only at
under Bibliography at the very end? Such a feature would not appear logical
in the context of Tufte's book template, would it?
The message from Tufte's book template seem to be: do not overcrowd with citations on any given page,
have sufficient text to justify the introduction (and context) of any new citation ... This criterion can be considered different for books when compared to peer-reviewed article ...


This text has been created with the command \verb+\lipsum[4]+.
\lipsum[1]

Relationships between results reported 
in\cite[-5ex]{Lib-OPUS-labs-1982-IEEE_TIT-Golay,Lib-OPUS-labs-1985-Phillips-Beenker}
and\cite[3ex]{Lib-OPUS-labs-1987-JourPhys-Bernasconi,Lib-OPUS-labs-1990-IEEE_TIT-Golay-skewsym}, 
and all subsequent updates under\cite[3ex]{Lib-OPUS2-labs-2014-homepage}
are depicted in four panels in Figure~\ref{fg-R-labs-wide-4-figures} (See Chapter~\ref{chap-Figures}.
The latest experimental results support the trend towards the conjectured asymptotic value of $F=12.3248$,
however as we demonstrate later on in the paper, the computational cost to reach this value may well exceed the currently available resources unless a better solver is discovered.


\section{About Text Items from Other Sources}
\label{chap-About-Other}

\newthought{The primary text items} for this section are selections
from\cite[-1ex]{Lib-OPUS2-labs-2015-arxiv-Boskovic}.

\vspace*{1.5ex}\noindent
The command \verb+\cmt+ as listed below,
\cmt{creates a `comment' sentence like this one.}\\  
\verb+\newcommand{\cmt}[1]{\textsf{[#1]}}+

\vspace*{1.5ex}\noindent
The command \verb+\OMIT+ as listed below\\
\verb+\newcommand{\OMIT}[1]{}+ \\
suppresses a block of text listed in the latex source code below
(it makes it invisible).
\OMIT{
\TOPIC{Local paragraph} 
This boldface {\em Local paragraph.} has been created by using 
the local command \verb+\TOPIC+: see the command below\\
\verb+\newcommand{\TOPIC}[1]{\vspace{1.3ex}\par\noindent\textbf{#1.}}+
}

\TOPIC{Local paragraph} 
This boldface headings, terminated with a period, {\em Local paragraph.} has been created by using 
the local command \verb+\TOPIC+: see the command below\\
\verb+\newcommand{\TOPIC}[1]{\vspace{1.3ex}\par\noindent\textbf{#1.}}+\\
\noindent
NOTE: such heading may be considered `too bold' in the context of Tufte's ideas in
{\em Beautiful Evidence}.

\paragraph{Tufte's Paragraph} Paragraph headings (as shown here) are introduced by
italicized text and separated from the main paragraph by a bit of space.
The command is \\
\verb+\paragraph{}+


\TOPIC{About the labs problem}
The {\em aperiodic low-autocorrelation binary sequence} (\labs) problem 
has a simple formulation: take a binary sequence of length $L$,  
$S = s_1 s_2 \ldots s_L$, $s_i \in \{ +1,-1 \}$, the autocorrelation function
$C_k(S) = \sum_{i=1}^{L-k}s_{i}s_{i+k},$ and minimize the energy function:
\begin{equation}
E(S) =  \sum_{k=1}^{L-1}C_{k}^{2}(S)
\label{eq_energy}
\end{equation}
or alternatively, maximize the 
{\em merit factor F}\cite[-1ex]{Lib-OPUS-labs-1977-IEEE_TIT-Golay,
Lib-OPUS-labs-1982-IEEE_TIT-Golay,
Lib-OPUS-labs-1990-IEEE_TIT-Golay-skewsym}:
\begin{equation}
F(S) =  {L^2}/({2E(S)}).
\label{eq_meritFactor}
\end{equation}
 

The asymptotic value for the maximum merit factor $F$, introduced
by Golay, has been re-derived
using arguments from statistical mechanics\cite[2ex]{Lib-OPUS-labs-1987-JourPhys-Bernasconi}:
\begin{equation}
{\rm as~~~} L \rightarrow \infty {\rm ,~~~then~~~}F  \rightarrow 12.3248
\label{eq_asymptotic}
\end{equation} 
The publication of the asymptotic value in Eq. \ref{eq_asymptotic}
is providing an on-going challenge since no published solutions can yet claim to converge to this value
as the length of the sequence increases. 

\TOPIC{Creating a filler text}
The remainder of this paragraph has been created with the command \verb+\lipsum[4]+.
\lipsum[4]

%\paragraph{Creating more filler text}
%The remainder of this paragraph has been created with command \verb+\lipsum[1]+.
%\lipsum[1]

\TOPIC{More about the labs problem}
Finding the optimum sequence is significantly harder than solving the special cases of the Ising spin-glass problems with limited interaction and periodic boundary conditions, for 
example\cite{Lib-OPUS-labs-2003-GECCO-Goldberg-periodic}. 
While effective methods have been presented to solve the special 
cases\cite[3ex]{Lib-OPUS-labs-2003-GECCO-Goldberg-periodic}, up to $L = 400$, 
the best merit factors that has also been {\em proven optimal} for the problem as formulated in Eq. \ref{eq_meritFactor}
are presently known for values of $L \le 60$ 
only\cite[3ex]{Lib-OPUS-labs-1996-JPhysA-Mertens-BB_solutions}.
A web page of \labs\ best merit factors and solutions, 
up to the sequence length of $L=304$, has been compiled by Joshua Knauer in 2002.
This page is no longer accessible and has now been restored 
at two mirroring sites\cite[3ex]{Lib-OPUS2-labs-2014-homepage-Knauer} next to 
additional and  comprehensive tables of {\em best-value solutions}.
These tables contain not only updates on the best known figures of merit but also 
on the number of {\em unique}  solutions in {\em canonic form} and the solutions themselves.
%

{\bf Now, we need more text on this page if we are to make
extra space for the citation that should be moved into the margins on the next page.}
Can or should we have {\em some citations} not appear in the margin, only at
under Bibliography at the very end? Such a feature would not appear logical
in the context of Tufte's book template, would it?
The message from Tufte's book template seem to be: do not overcrowd with citations on any given page,
have sufficient text to justify the introduction (and context) of any new citation ... This criterion can be considered different for books when compared to peer-reviewed article ...


This text has been created with the command \verb+\lipsum[4]+.
\lipsum[1]

Relationships between results reported 
in\cite[-5ex]{Lib-OPUS-labs-1982-IEEE_TIT-Golay,Lib-OPUS-labs-1985-Phillips-Beenker}
and\cite[3ex]{Lib-OPUS-labs-1987-JourPhys-Bernasconi,Lib-OPUS-labs-1990-IEEE_TIT-Golay-skewsym}, 
and all subsequent updates under\cite[3ex]{Lib-OPUS2-labs-2014-homepage}
are depicted in four panels in Figure~\ref{fg-R-labs-wide-4-figures} (See Chapter~\ref{chap-Figures}.
The latest experimental results support the trend towards the conjectured asymptotic value of $F=12.3248$,
however as we demonstrate later on in the paper, the computational cost to reach this value may well exceed the currently available resources unless a better solver is discovered.

