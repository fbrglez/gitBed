
\begin{figure*}[t!]
\vspace*{-1ex}
%\subfloat[\lssMAts\ solver, based on $MA_{TS}$ in~\cite{Lib-OPUS-labs-2009-ASC-Gallardo-memetic}.]{
\begin{minipage}[t]{0.49\linewidth}
\centering
\begin{algorithmic}[1]
\PROCEDURE{\CALL{lssMAts}{$\Theta^{ub}_L, t_{lmt}$}}
\FOR{$i\gets 1$ {\bf to} $popsize$} \label{alg_lssMAts_For1}
    \STATE $pop_i\gets$ \CALL{RandomBinarySequence}{$L$}
    \STATE \CALL{Evaluate}{$pop_i$}
\ENDFOR \label{alg_lssMAts_For2}
\STATE \colorbox{Gray}{\valueBest $\gets \CALL{ValueBest}{pop}$}
\WHILE {\colorbox{Gray}{ $t < t_{lmt}$ {\bf ~and~}  \valueBest $\,>\,$ \valueTarget} }
    \FOR{$i=1$ {\bf to} $\mathit{offsize}$}     
        \IF {recombination is performed ($p_X$)} \label{alg_lssMAts_Rec1}
            \STATE $parent_1 \gets $\CALL{Select}{$pop$}
            \STATE $parent_2 \gets $\CALL{Select}{$pop$}
            \STATE $\mathit{offspring_i}\gets$\CALL{Recombine}{$parent_1, parent_2$}
        \ELSE
            \STATE $\mathit{offspring}_i \gets $\CALL{Select}{$pop$}
        \ENDIF \label{alg_lssMAts_Rec2}
        \IF {mutation is performed ($p_m$)} \label{alg_lssMAts_Mut1}
            \STATE $\mathit{offspring}_i \gets $\CALL{Mutate}{$\mathit{offspring}_i$}
        \ENDIF \label{alg_lssMAts_Mut2}
        \STATE $\mathit{offspring}_i \gets$\CALL{TabuSearch}{$\mathit{offspring}_i$}
        \STATE \CALL{Evaluate}{$\mathit{offspring}_i$}
    \ENDFOR
    \STATE $pop \gets$\CALL{Replace}{$pop$, $\mathit{offspring}$} \label{alg_lssMAts_Sel}
    \STATE \colorbox{Gray}{\valueBest $\gets \CALL{ValueBest}{pop}$}
\ENDWHILE
\ENDPROCEDURE 
\end{algorithmic}
\label{fg-lssMAts-lssRRts-a}
\end{minipage}
%}
%\subfloat[\lssRRts\ solver, based on reduction of \lssMAts.]{
\begin{minipage}[t]{0.49\linewidth}
{\small	
The procedure \lssMAts\ on the left is an instrumented versions of the 
\labs\ solver named as $MA_{TS}$
in~ cite {\em Lib-OPUS-labs-2009-ASC-Gallardo-memetic}.
Settings of all parameters,
used also in our experiments, are described
in~ cite {\em Lib-OPUS-labs-2009-ASC-Gallardo-memetic}.
See a concise reprise below.
\par\vspace*{2ex}
\begin{tabular}{l l}
 {\bf setting} & {\bf value} \\
 \hline
 population size: & 100 \\
 mutation probability: & $2/(L+1)$ \\
 crossover probability: & 0.9 \\
 tournament selection size: & 2 \\
 crossover: & uniform \\
 tabu search walk length: & a random choice  \\
 ~~              &  from the range\\
 ~~              & {\small [$\frac{L}{2}, \frac{3 L}{2}$]}
\end{tabular}
}
\vspace{0.6cm}
\centering
\begin{algorithmic}[1]
\PROCEDURE{\CALL{lssRRts}{$\Theta^{ub}_L, t_{lmt}$}}
\STATE $pop_1\gets$ \CALL{RandomBinarySequence}{$L$}
\STATE \CALL{Evaluate}{$pop_1$}
\STATE \colorbox{Gray} {\valueBest $\gets \CALL{ValueBest}{pop}$}
\WHILE {\colorbox{Gray}{ $t < t_{lmt}$ {\bf ~and~}  \valueBest $\,>\,$ \valueTarget} }
        \STATE $\mathit{pop_1} \gets $\CALL{RandomBinarySequence}{$L$}
        \STATE $\mathit{pop_1} \gets $\CALL{TabuSearch}{$\mathit{pop_1}$}
        \STATE \CALL{Evaluate}{$\mathit{pop_1}$}
        \STATE \colorbox{Gray}{\valueBest $\gets \CALL{ValueBest}{pop}$}
\ENDWHILE
\ENDPROCEDURE 
\end{algorithmic}
\vspace{0.18cm}
\label{fg-lssMAts-lssRRts-b}
\end{minipage}
%}
%% NOTE the short caption feature with []: [Two instrumented versions of the \labs\ solver $MA_{TS}$]
\caption[Figure file: fg-lssMAts-lssRRts.tex]{We illustrate two instrumented versions of the \labs\ solver named as $MA_{TS}$
in cite {\em Lib-OPUS-labs-2009-ASC-Gallardo-memetic}  
under the caption ``Pseudo code of the memetic algorithm''.} 
\label{fg-lssMAts-lssRRts}
\end{figure*}
 